\documentclass[12pt]{article}

\begin{document}

{\parindent=0pt \bf Cross-section:}
\vspace{0.5cm}

Cross-section calculated by {\it crosssection.f}, based on~\cite{Vogel:1999zy}.
Command line:

\begin{center}
  \framebox{gfortran -o c.exe crosssection.f -lmathlib}
\end{center}

When executed, create file {\it crosskaml.dat} with:
\[
\frac{d^2\sigma(E_\nu,T_{e^+})}{dE_\nu dT_{e^+}}
\]

\vspace{1cm}
{\parindent=0pt \bf Effective cross-section:}
\vspace{0.5cm}

Based on $L/E$ binning (fig.5 in~\cite{Gando:2010aa}), we calculate an
effective cross-section based on that particular energy binning with
program {\it espectrokamland.le.f}:
\begin{center}
  \framebox{gfortran -o e.exe espectrokamland.le.f -lmathlib -lkernlib}
\end{center}
where the following is calculated:
\[
\frac{d\sigma(E_\nu)}{dE_\nu}=
\int_{\tilde{T}_{e^+}^{min}}^{\tilde{T}_{e^+}^{max}} d\tilde{T}_{e^+}
\int dT_{e^+} R(T_{e^+},\tilde{T}_{e^+})
\frac{d^2\sigma(T_\nu,T_{e^+})}{dT_\nu dT_{e^+}}
\]
where
\[
R(T_{e^+},\tilde{T}_{e^+})=\frac{1}{\sqrt{2\pi\sigma_T^2}}
\exp{\left[-\left(\frac{\tilde{T}_{e^+}-T_{e^+}}{2\sigma_T}\right)^2\right]}
\]
with
\[
\sigma_T=0.065\sqrt{T_{e^+}}
\]

\vspace{1cm}
{\parindent=0pt \bf Statistical Analysis:}
\vspace{0.5cm}

The $\chi^2$ analysis is performed with routine {\it chi2.f} and subroutines
{\it subchi2.f} and {\it subspec.f}:
\begin{center}
  \framebox{gfortran -o c.exe chi2.f subchi2.f subspec.f -lmathlib}
\end{center}
which calculates the rates and $\chi^2$ of KamLAND. We used the following
expression for the $\chi^2$, based on Poisson statystics:
\[
\chi^2=2\sum_i\left[fR_i^{th}-R_i^{exp}+
R_i^{exp}log\left(\frac{R_i^{exp}}{fR_i^{th}}\right)\right] +
\left(\frac{1-f}{0.041}\right)^2
\]
where the $\chi^2$ is minimized in the total flux normalization factor $f$.

\vspace{1cm}
{\parindent=0pt \bf Flux and Data:}
\vspace{0.5cm}

The Flux is calculated through the following parameterization:
\[
\phi=\sum_{j=1,4}\phi_j\exp{\left(a_{0,j}+a_{1,j}E_\nu+a_{2,j}E_\nu^2\right)}
\]
where the factors $a$ and $\phi_i$ are provided by KamLAND, table~\ref{tab:nuflux}.

\begin{table}{h!}
  \begin{center}
    \caption{Neutrino flux parameters}
    \label{tab:nuflux}
    \begin{tabular}{l|c|c|c|c}
      \hline
      $\phi_i$ & 0.570 & 0.295 & 0.078 & 0.057\\
      $a_0$ & 0.870 & 0.896 & 0.976 & 0.793\\
      $a_1$ &-0.160 &-0.239 &-0.162 &-0.080\\
      $a_2$ &-.0910 &-0.0981&-0.079 &-0.1085\\
      \hline
    \end{tabular}
  \end{center}
\end{table}

We assume the same flux composition for all reactors. We use 22 reactors with the distances and contributions to total flux in KamLAND given by table~\ref{tab:reactors}
        
\begin{table}{h!}
  \begin{center}
    \caption{Reactors}
    \label{tab:reactors}
    \begin{tabular}{l|c|c}
      \hline
      name & contribution (\%) & distance (km)\\
      &30.9 &160 \\
      &13.8 &179 \\
      &9.0  &191 \\
      &7.9  & 88 \\
      &7.6  &138 \\
      &7.5  &214 \\
      &7.4  &146 \\
      &3.8  &349 \\
      &3.5  &351 \\
      &1.3  &141 \\
      &1.2  &295 \\
      &0.9  &138 \\
      &0.8  &401 \\
      &0.7  &431 \\
      &0.6  &561 \\
      &0.4  &754 \\
      &0.2  &830 \\
      &0.2  &783 \\
      &0.8  &712 \\
      &0.6  &735 \\
      &0.5  &709 \\
      &0.5  &986 \\
      \hline
    \end{tabular}
  \end{center}
\end{table}

\begin{thebibliography}{99}

%\cite{Vogel:1999zy}
\bibitem{Vogel:1999zy} 
  P.~Vogel and J.~F.~Beacom,
  ``Angular distribution of neutron inverse beta decay, anti-neutrino(e) +
  p $\rightarrow$ e+ + n,''
  Phys.\ Rev.\ D {\bf 60}, 053003 (1999)
  doi:10.1103/PhysRevD.60.053003
  [hep-ph/9903554].
  %%CITATION = doi:10.1103/PhysRevD.60.053003;%%
  %462 citations counted in INSPIRE as of 14 Aug 2018

%\cite{Gando:2010aa}
\bibitem{Gando:2010aa} 
  A.~Gando {\it et al.} [KamLAND Collaboration],
  ``Constraints on $\theta_{13}$ from A Three-Flavor Oscillation Analysis of Reactor Antineutrinos at KamLAND,''
  Phys.\ Rev.\ D {\bf 83}, 052002 (2011)
  doi:10.1103/PhysRevD.83.052002
  [arXiv:1009.4771 [hep-ex]].
  %%CITATION = doi:10.1103/PhysRevD.83.052002;%%
  %297 citations counted in INSPIRE as of 14 Aug 2018

\end{thebibliography}

\end{document}

